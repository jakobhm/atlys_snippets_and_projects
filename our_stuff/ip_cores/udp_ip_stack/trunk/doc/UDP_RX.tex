\chapter{UDP/IP-Stack}

\section{UDP_RX}

\subsection{UDP_RX - Entity}
Es wird folgende Entity definiert.
	\begin{lstlisting}[style=myVHDL]
	entity UDP_RX is
	  port (
		 -- UDP Layer signals
		 udp_rx_start : out std_logic;       -- indicates receipt of udp header
		 udp_rxo      : out udp_rx_type;
		 -- system signals
		 clk          : in  std_logic;
		 reset        : in  std_logic;
		 -- IP layer RX signals
		 ip_rx_start  : in  std_logic;       -- indicates receipt of ip header
		 ip_rx        : in  ipv4_rx_type
		 );                  
	end UDP_RX;
	\end{lstlisting}

Die Signale udp_rx_start und udp_rxo werden anschließend aus UDP_Complete_nomac herausgeführt.
ip_rx_start und ip_rx bediehnen das IP_Complete_nomac module.
clk und reset stehen senktrecht dazu und dienen als Steuersigale.

In udp_rxo verstecken sich ein Byte der empfangenen Daten.
In hdr befindet sich der Header des UDP-Frames mit den Ports der Quell-IP-Adresse, mit der Gültigkeit und der Datenfeldlänge. 
	\begin{lstlisting}[style=myVHDL]
	-- Aus der Datei ipv4_types.vhdl
 	type udp_rx_type is record
		hdr				: udp_rx_header_type;						-- header received
		data				: axi_in_type;									-- rx axi bus
	end record;

	-- Aus der Datei axi.vhdl

	type axi_in_type is record
		data_in 				: STD_LOGIC_VECTOR (7 downto 0);
		data_in_valid 		: STD_LOGIC;								-- indicates data_in valid on clock
		data_in_last 		: STD_LOGIC;								-- indicates last data in frame
	end record;
	\end{lstlisting}

Im ipv4_rx_type werden die Daten von der IP-Schicht entgegengenommen. 
	\begin{lstlisting}[style=myVHDL]
	-- Aus der Datei ipv4_types.vhdl
	type ipv4_rx_type is record
		hdr				: ipv4_rx_header_type;						-- header received
		data				: axi_in_type;									-- rx axi bus
	end record;

	-- Aus der Datei ipv4_types.vhdl
	type ipv4_rx_header_type is record
		is_valid				: std_logic;
		protocol				: std_logic_vector (7 downto 0);
		data_length			: STD_LOGIC_VECTOR (15 downto 0);	-- user data size, bytes
		src_ip_addr 		: STD_LOGIC_VECTOR (31 downto 0);
		num_frame_errors	: std_logic_vector (7 downto 0);
		last_error_code	: std_logic_vector (3 downto 0);		-- see RX_EC_xxx constants
		is_broadcast		: std_logic;								-- set if the msg received is a 
																				-- broadcast
	end record;

	\end{lstlisting}
Im hdr befindet sich der IP-Header.

\subsection{UDP_RX - Architecture}
Besteht aus einer Finitstatemachine. 
Es werden fünf States definiert: 
	\begin{lstlisting}[style=myVHDL] 
	type rx_state_type is (IDLE, UDP_HDR, USER_DATA, WAIT_END, ERR);
   type rx_event_type is (NO_EVENT, DATA);

	-- wichtige Signale
	  -- state variables
	  signal rx_state         : rx_state_type;
	  signal rx_count         : unsigned (15 downto 0);
	  signal src_port         : std_logic_vector (15 downto 0);  -- src port captured from input
	  signal dst_port         : std_logic_vector (15 downto 0);  -- dst port captured from input
	  signal data_len         : std_logic_vector (15 downto 0);  -- user data length captured from
	
	  -- rx control signals
	  signal next_rx_state    : rx_state_type;
	  signal set_rx_state     : std_logic;
	  signal rx_event         : rx_event_type;
	\end{lstlisting}

Neben der Statevariable rx_state ist hier rx_event sehr wichtig. 
Mit dieser wird der Statmachine signalisiert ob ein Byte im Empfangsbyte ist.

































