\section{IP-IPv4}
\paragraph{Allgemeines}
Das IP befindet sich auf der 3. Schicht (Vermittlung/Network) des OSI-Schichtenmodels und auf Schicht 2 des TCP/IP-Schichtenmodels.
Es gibt das IPv4 und das IPv6. Mit dem IPv6 wurde der Umfang der IP-Adressen um 96bit erweitert also auf 128bit im Gegensatz zu dem IPv4-Protokoll mit 32bit. 
Bei unserem IP/UDP-Opensource Stack wird das IPv4-Protokoll verwendet.
Deshalb wird diese im speziellen betrachtet und mit IP bezeichnet. 
Üblicherweise werden die 4 Bytes mit Punkten getrennt in dezimaler Schreibweise dargestellt. 
Wie z.B. 207.142.131.235. FÜr die Vergabe der Nummern und Namen im Internet ist IANA zuständig. 
Die IP-Adressen werden in Netzklassen eingeteilt. 


\subsection{Paket}
Die Paketlänge eines IP-Pakets kann maximal 65535 Byte lang sein (2¹⁶ - 1). 
Hierbei sind 20 Byte für den IP-Header reserviert.
Deshalb stehen maximal 65515 Byte für das darüberliegende Protokoll zur Verfügung.
Normalerweise wird das Paket auf die maximale Länge beschrängt, die der darunterliegende Frame bietet. 
Bei Ethernet wäre dies 1500 Bytes Daten.
Davon gehen 20 für den IP-Header flöten.
Also ist eine wichtige Rechengröße für darüberliegende Protokolle die 1480 Byte.
Wenn eine IP-Frame über Ethernet versendet wird muss dieses mindestens 64 Bytes groß sein. 
Ist der Frame kleiner wird er durch Nullbytes auf diese Länge erweitert.
Im length-Feld, des IP-Pakets, kann die wahre Länge ermittelt werden.
Im Ethernet-Netzwerk haben alle Karten eine 48-Bit-Adresse. 
Eine Kommunikation über IP-Paketen von einer Karte zur nächsten kann nur über diese MAC-Adresse erfolgen.
Also besitzt jedes IP-Sendebereites Ethernet-System eine IP-Adressen und eine MAC-Adressen.
Über das ARP (Address Resolution Protocol) kann ein System sich bei bekannter MAC- bzw. die IP-Adresse die fehlende Unbekannte eines weiteren IP-Systems erfragen.
Die IP/MAC-Adresszuordung wird in jedem Rechner im ARP-Cache gespeichert und verwaltet.
Über ARP-Request können unbekannte Adressen abgefragt werden und werden durch ARP-Replys beantwortet.
Viele Optionen des IPv4-Headers bleiben ungenutzt. 
Der IP-Header hat eine Prüfsumme, die aber nur über den IP-Header gebildet wird. Über höhere Protokolle wie UDP kann dem Datenfeld eine Prüfsummen hinzugefügt werden. 
Außerdem fügt das TCP bzw. UDP noch Portadressen hinzu die für die als Prozessadressen bzw. als Endpunkte fungieren.
TCP baut eine Verbindung zwischen diesen Endpunkten/Portadressen auf und kommuniziert somit mit anderen im Netz entfernten Prozessen.
Bei der Übertragung können Pakete im Netzwerk fragmentiert werden.
Hierbei wird durch Flags angezeigt, dass dem gerade Empfangenem Paket noch weitere Fragmente folgen.
Im Fragment Offset Feld wird der Ort des Fragments im uhrsprünglichem Header beschrieben.
Folgt kein Fragment mehr dem IP-Paket mit dem höchsten Fragment Offset ist das MF-Flag im Header gesetzt.

\section{ARP - Adress Resolution Protocol }
ARP kann in unterschiedlichen Netzwerken verwendet werden.
Im Ethernet besteht zwischen IP- und MAC-Adressen kein fester Zusammenhang.
Außerdem kann im Ethernet nur über die MAC-Adresse einem Rechner ein Ethernet-Frame zugestellt werden indem sich ein IP-Packet befindet. 
Bei der Zustellung spielt die IP-Adresse keine Rolle.
Hat man nun aber nur die IP-Adresse braucht man eine Möglichkeit auf die dazugehörige MAC-Adresse schließen zu können.  
Um dieses Problem aus dem Weg zu schaffen wurde ARP etwickelt.
ARP wird fast ausschließlich in Ethernet-Netzen verwendet und dient der Zuordnung von MAC- zu IP- bzw IP- zu MAC-Adressen.
In IPv6 wird das von NDP erledigt.
Die Zuordung wird in ARP-Tabellen der beteiligten Rechner hinterlegt.
 
\subsection{Funktionsweise}
Ein Rechner ordnet einer bekannten IP-Adresse eine MAC-Adresse zu.
Dazu wird ein ARP-Anforderung in dem die eigene MAC- und IP-Adresse und die IP-Adresse des gesuchten Computers verpackt ist vesendet. 
Die MAC-Adresse des gesuchten Rechners wird im ARP-Header auf die Broadcast-Adresse ff-ff-ff-ff-ff-ff gesetzt. 
Diese ARP-Paket mit der Broadcast-Adresse wird von jedem PC aufgenommen.
Der Empfänger der im Header seine IP-Adresse wiederfindet ersetzt die Broadcast-Adresse durch seine MAC-Adresse und schickt das Paket zurück (ARP-Reply).
Empfängt der anfragende Rechner diese Packet, schreibt er die MAC- und IP-Adresse in den ARP-Cache.
Der angefragte Rechner speichert meist auch die MAC- und IP-Adresse des fragenden Rechners.
Der ARP-Cache speichert Protokolltyp, Protokolladresse des Senders, Hardware-Adresse des Senders und Eintragszeitpunkt.
Der Eintragszeitpunkt gibt an wann ein Eintrag als ungültig erklärt wird und neu angefordert werden muss.
Unter Unix und Windows kann mit "arp -a" der Cache angezeigt werden.
Das ARP arbeitet nur auf lokaler Netzwerkebene.

\paragraph{ARP - Paketformat}
Das ARP-Paket wird im Ethernetframe versendet.
Das Typfeld des Ethernetframes wird auf 0x0806 gesetzt. 
0x0806 ist für ARP-Pakete reserviert und zeichnet es als solches aus.
Da ARP-Pakete kurtz sind werden sie mit zusätzlichen Nullen auf die minmale Framgröße von 64 Byte aufgefüllt.
Durch die Adresstypen und Adresslängen könnte Theoretisch das ARP-Protokoll auch für z.B. das IPv6 eingesetzt werden.
Dieses verwendet aber das NDP.

\subsection{Spezielle ARP-Nachrichten}

\paragraph{Proxy ARP}
Es sind zwei Rechner durch einen Router getrennt.
Der anfragende Rechner sendet eine Anfrage an Rechner 2.
Hier antwortet der Router auf die IP-Adresse, aber mit seiner Router-MAC-Adresse die in Richtung anfragenden Rechner schaut.
Somit sendet der anfragende Rechner die Pakete zuerst an den Router der die Pakete dann weiterleitet.

\paragraph{Gratuitous ARP}
Bei dieser Art sendet der Rechner ein ARP-Paket in dem seine IP-Adresse sowohl als Sende- als auch als die gesuchte Zieladresse eingetragen ist.

\pargraph{RARP - Reverse-ARP}
Hierbei wird nicht die MAC- sonder die IP-Adresse gesucht.

\section{UDP}
UDP siedelt sich über dem IP an auf der 3. Ebene (Transport) des TCP/IP-Protokollstapels.
Im ISO/OSI-Model auf der 4. Schicht (Transport).
Mit UDP bzw. TCP kann über die Portadresse das Paket von Anwendung via Netz zu einer anderen Anwendung verschickt werden.
Dies nennt man Anwendungsmultiplexen bzw. -demultiplexen.
Somit ist die "Adress-Reihenfolge" von unten nach oben wie folgt: MAC-Adresse, IP-Adresse, Portadresse.
Außerdem fügt UDP quasi dem Datenfeld des IP-Pakets eine Prüfsumme hinzu.
Die Prüfsumme des UDP wird über das komplette Paket gebildet.
Die größe des übertragbaren Daten in einem UDP-Datenbereich ist theoretisch 2¹⁶-1.
Diese Menge wird dann jedoch fragmentiert im IP-Paket übertragen, dass selbst nur maximal 65515 Bytes im Datenfeld übertragen kann.
Somit kann auf einmal ein Daten-Paket von 65535-8(UDP-Header) übertragen werden.

\section{unser UDP/IP-Stack}

\paragraph{ was wird nicht unterstüzt }
Es ist keine Fragmentierung der Pakete möglich.

\begin{itemize}
  \item keine Fragmentierung des IP-Pakets
  \item 
  \item The third etc \ldots
\end{itemize}

